%%%%%%%%%%%%%%%%%%%%%%%%%%%%%%%%%%%%%%%%%%%%%%%%%%%%%%%%
\fondo{celeste}
\section{Caso de uso}
\fondo{blanco}
%%%%%%%%%%%%%%%%%%%%%%%%%%%%%%%%%%%%%%%%%%%%%%%%%%%%%%%%

%%%%%%%%%%%%%%%%%%%%%%%%%%%%%%%%%%%%%%%%%%%%%%%%%%%%%%%%
\begin{frame}
    \frametitle{Caso de uso}

    \begin{itemize}
        \item \textbf{Asignatura:} Álgebra Lineal
        \item \textbf{Carrera:} Ciencia de Datos
        \item \textbf{Nivel:} Segundo nivel
        \item \textbf{Resultado de aprendizaje:} Resuelve operaciones con matrices, incluyendo productos, cálculo de inversas; además del  cálculo de determinantes de matrices, valores y vectores propios y su significado en el contexto del Álgebra Lineal.
    \end{itemize}
\end{frame}
%%%%%%%%%%%%%%%%%%%%%%%%%%%%%%%%%%%%%%%%%%%%%%%%%%%%%%%%

%%%%%%%%%%%%%%%%%%%%%%%%%%%%%%%%%%%%%%%%%%%%%%%%%%%%%%%%
\begin{frame}
\centering
    \postitimg[0.6\linewidth]{Figuras/Fig07.png}
\end{frame}
%%%%%%%%%%%%%%%%%%%%%%%%%%%%%%%%%%%%%%%%%%%%%%%%%%%%%%%%

%%%%%%%%%%%%%%%%%%%%%%%%%%%%%%%%%%%%%%%%%%%%%%%%%%%%%%%%
\begin{frame}
    \frametitle{Adquisición de concepto}

    \small

    Para la adquisición del concepto, se solicitará al estudiante interactuar con ChatGPT y la visualización de video, siguiendo los siguientes pasos:

\begin{enumerate}[leftmargin=*,label=\arabic*.]
    \item Interactuar con ChatGPT mediante los siguientes \textit{prompts}, leyendo detenidamente el \textit{prompt} y su respuesta:
    \begin{enumerate}[label=\textit{Prompt \arabic*.},leftmargin=2.1cm]
        \item Vas a ser mi profesor de la asignatura de Álgebra Lineal, te iré dando indicaciones y me irás explicando de manera formal lo que te pida. Vas a tener mucho cuidado al escribir la parte matemática para que se visualice bien. Sé divertido. ¿Entendido?
        \item ¿Cómo se calcula un valor propio de una matriz? No me des un ejemplo numérico aún.
        \item Dame un ejemplo del cálculo de valores propios con una matriz de 2 por 2.
    \end{enumerate}
\end{enumerate}
\end{frame}
%%%%%%%%%%%%%%%%%%%%%%%%%%%%%%%%%%%%%%%%%%%%%%%%%%%%%%%%

%%%%%%%%%%%%%%%%%%%%%%%%%%%%%%%%%%%%%%%%%%%%%%%%%%%%%%%%
\begin{frame}
    \frametitle{Adquisición de concepto}
\small
\begin{enumerate}[leftmargin=*,label=\arabic*.,start=2]
    \item Visualiza el siguiente video: \href{https://youtu.be/HET8XcIX-n4?si=t4lUbTmWaPOTbtAM}{Obteniendo los valores propios de una matriz de 2$\times$2}.
    \item Continúa la interacción con ChatGPT mediante los siguientes \textit{prompts}, leyendo detenidamente el \textit{prompt} y su respuesta:
    \begin{enumerate}[label=\textit{Prompt \arabic*.},leftmargin=2.1cm,start=4]
        \item Dame un ejemplo del cálculo de valores propios con una matriz de 3 por 3, que el ejemplo sea en una matriz triangular. Realízalo paso a paso con el cálculo de determinante.
        \item Plantéame un ejercicio de cálculo de valores propios en matrices de 2 por 2.
    \end{enumerate}
    \item Visualiza el video: \href{https://youtu.be/Gx0PaWI9eYo?si=oTPRSIfeEopspelW}{Vectores propios y valores propios}.
\end{enumerate}
\end{frame}
%%%%%%%%%%%%%%%%%%%%%%%%%%%%%%%%%%%%%%%%%%%%%%%%%%%%%%%%

%%%%%%%%%%%%%%%%%%%%%%%%%%%%%%%%%%%%%%%%%%%%%%%%%%%%%%%%
\begin{frame}
    \frametitle{Adquisición de concepto}
\small
\begin{enumerate}[leftmargin=*, label=\arabic*., start=4]
    \item Continúa la interacción con ChatGPT con las preguntas sobre el video que acabas de ver.
     \item En caso de tener más dudas sobre el tema, interactúa con tus compañeros de clase para solventarlas.
    \item Realiza el cuestionario del aula virtual.
\end{enumerate}
\end{frame}
%%%%%%%%%%%%%%%%%%%%%%%%%%%%%%%%%%%%%%%%%%%%%%%%%%%%%%%%

%%%%%%%%%%%%%%%%%%%%%%%%%%%%%%%%%%%%%%%%%%%%%%%%%%%%%%%%
\begin{frame}
    \begin{itemize}
        \item \textbf{Personalización de la actividad:}
        \begin{itemize}
            \item Estudiantes continúan interactuando con ChatGPT hasta asimilar completamente el concepto.
        \end{itemize}
        \item \textbf{Solventación de dudas:}
        \begin{itemize}
            \item Uso de ChatGPT para resolver dudas específicas sobre el tema y trabajo colaborativo con compañeros.
        \end{itemize}
        \item \textbf{Micro-tarea:}
        \begin{itemize}
            \item Estudiantes deben copiar el enlace del chat con ChatGPT como evidencia.
            \item Completar el cuestionario en el aula virtual.
        \end{itemize}
    \end{itemize}
\end{frame}
%%%%%%%%%%%%%%%%%%%%%%%%%%%%%%%%%%%%%%%%%%%%%%%%%%%%%%%%

%%%%%%%%%%%%%%%%%%%%%%%%%%%%%%%%%%%%%%%%%%%%%%%%%%%%%%%%
\begin{frame}
\centering
    \postitimg[0.75\linewidth]{Figuras/Fig08.png}\\
    {\footnotesize\url{https://chatgpt.com/share/59b3c97a-23da-4a56-8447-c23649c6f868}}
\end{frame}
%%%%%%%%%%%%%%%%%%%%%%%%%%%%%%%%%%%%%%%%%%%%%%%%%%%%%%%%

%%%%%%%%%%%%%%%%%%%%%%%%%%%%%%%%%%%%%%%%%%%%%%%%%%%%%%%%
\begin{frame}
\centering
    \postitimg[0.7\linewidth]{Figuras/Fig09.png}
\end{frame}
%%%%%%%%%%%%%%%%%%%%%%%%%%%%%%%%%%%%%%%%%%%%%%%%%%%%%%%%

%%%%%%%%%%%%%%%%%%%%%%%%%%%%%%%%%%%%%%%%%%%%%%%%%%%%%%%%
\begin{frame}
    \frametitle{Cuestionario en aula virtual}
    \begin{itemize}
        \item Copia el enlace del chat con ChatGPT como evidencia de la actividad realizada en casa.
        \item ¿Alguna pregunta que ChatGPT no te supo responder?
        \item En caso de que algún compañero te haya ayudado a resolver tus dudas, indica aquí quién o quienes te ayudaron.
        \item ¿Qué dudas tienes sobre calcular los valores propios de una matriz?
    \end{itemize}
\end{frame}
%%%%%%%%%%%%%%%%%%%%%%%%%%%%%%%%%%%%%%%%%%%%%%%%%%%%%%%%

%%%%%%%%%%%%%%%%%%%%%%%%%%%%%%%%%%%%%%%%%%%%%%%%%%%%%%%%
\begin{frame}
    \frametitle{Tareas en Clase y Evaluación}
    \begin{itemize}[leftmargin=*]
        \item \textbf{Visión conjunta:}
        \begin{itemize}
            \item Conectar actividades en casa con tareas en clase, enfocadas en el cálculo de vectores propios.
        \end{itemize}
        \item \textbf{Retroalimentación:}
        \begin{itemize}
            \item Feedback sobre las respuestas de la micro-tarea.
        \end{itemize}
        \item \textbf{Actividad de aplicación:}
        \begin{itemize}
            \item Ejercicios de cálculo de valores propios con matrices de diferentes tamaños.
        \end{itemize}
        \item \textbf{Micro-evaluación:}
        \begin{itemize}
            \item No se realiza una micro-evaluación específica en esta actividad.
        \end{itemize}
    \end{itemize}
\end{frame}
%%%%%%%%%%%%%%%%%%%%%%%%%%%%%%%%%%%%%%%%%%%%%%%%%%%%%%%%
