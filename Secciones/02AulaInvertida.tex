%%%%%%%%%%%%%%%%%%%%%%%%%%%%%%%%%%%%%%%%%%%%%%%%%%%%%%%%
\fondo{celeste}
\section{¿Qué es la Metodología de Aula Invertida?}
\fondo{blanco}
%%%%%%%%%%%%%%%%%%%%%%%%%%%%%%%%%%%%%%%%%%%%%%%%%%%%%%%%


%%%%%%%%%%%%%%%%%%%%%%%%%%%%%%%%%%%%%%%%%%%%%%%%%%%%%%%%
\begin{frame}
    \frametitle{¿Qué es la Metodología de Aula Invertida?}
    \vspace{3mm}
    \begin{columns}
    \column{.4\textwidth}
    \begin{itemize}[leftmargin=*]
        \item Los estudiantes acceden a los contenidos antes de la clase.
        \item El tiempo en clase se dedica a actividades prácticas y resolución de problemas.
    \end{itemize}
    \column{.6\textwidth}
        \postitimg[0.99\linewidth]{Figuras/Fig02.png}
    \end{columns}
\end{frame}
%%%%%%%%%%%%%%%%%%%%%%%%%%%%%%%%%%%%%%%%%%%%%%%%%%%%%%%%


%%%%%%%%%%%%%%%%%%%%%%%%%%%%%%%%%%%%%%%%%%%%%%%%%%%%%%%%
\begin{frame}
    \frametitle{Particularidades del Aula Invertida}
    \begin{columns}
    \column{.5\textwidth}
    \begin{block}{Preclase:}
        \begin{itemize}
            \item Adquisición de conceptos.
            \item Personalización del contenido.
            \item Resolución  de dudas.
            \item Micro-tarea.
        \end{itemize}
    \end{block}
    \column{.5\textwidth}
    \begin{block}{En Clase:}
        \begin{itemize}
            \item Visión conjunta.
            \item Retroalimentación.
            \item Actividad de aplicación.
            \item Micro-evaluación.
        \end{itemize}
    \end{block}
    \end{columns}
\end{frame}
%%%%%%%%%%%%%%%%%%%%%%%%%%%%%%%%%%%%%%%%%%%%%%%%%%%%%%%%

%%%%%%%%%%%%%%%%%%%%%%%%%%%%%%%%%%%%%%%%%%%%%%%%%%%%%%%%
\begin{frame}
    \frametitle{Beneficios del Aula Invertida}
    \vspace{-5mm}
    \begin{columns}
    \column{.5\textwidth}
        \begin{center}
        \postitimg[0.99\linewidth]{Figuras/Fig03.jpg}
        \end{center}
    \column{.5\textwidth}
        \begin{itemize}[leftmargin=*]
            \item Mayor participación activa de los estudiantes.
            \item Aprendizaje más profundo y significativo.
            \item Flexibilidad en el ritmo de aprendizaje.
            \item Fomento del pensamiento crítico y la colaboración.
        \end{itemize}
    \end{columns}
\end{frame}
%%%%%%%%%%%%%%%%%%%%%%%%%%%%%%%%%%%%%%%%%%%%%%%%%%%%%%%%

%%%%%%%%%%%%%%%%%%%%%%%%%%%%%%%%%%%%%%%%%%%%%%%%%%%%%%%%
\begin{frame}
    \frametitle{Desafíos del Aula Invertida}
    \begin{columns}
    \column{.5\textwidth}
        \begin{itemize}[leftmargin=*]
            \item Resistencia al cambio por parte de estudiantes y docentes.
            \item Requiere una mayor preparación previa.
            \item Necesidad de acceso a recursos tecnológicos.
            \item Implementar mecanismos de evaluación continua.
        \end{itemize}
    \column{.5\textwidth}
        \begin{center}
        \postitimg[0.99\linewidth]{Figuras/Fig04.jpg}
        \end{center}
    \end{columns}
\end{frame}
%%%%%%%%%%%%%%%%%%%%%%%%%%%%%%%%%%%%%%%%%%%%%%%%%%%%%%%%
